%% aidisclose-doc.tex --- Utilities for Generative AI disclosure checklist and statements
%%
%% Copyright (C) 2025 by João M. Lourenço
%%
%% This file may be distributed and/or modified under the conditions of
%% the LaTeX Project Public License, either version 1.3c of this license
%% or (at your option) any later version. The latest version of this
%% license is in:
%%
%%    http://www.latex-project.org/lppl.txt
%%
%% and version 1.3c or later is part of all distributions of LaTeX
%% version 2006/05/20 or later.
%%
\documentclass[11pt]{article}


% \usepackage[portuguese]{babel}

%%%%%%%%%%%%%%%%%%%%%%%%%%%%%%%%%%%%%%%%%%
%% 1. Basic Utilities & Formatting
\usepackage[T1]{fontenc}
\usepackage{xcolor}      % Must be before hyperref (for colors)
\usepackage{url}
\usepackage{booktabs}

%%%%%%%%%%%%%%%%%%%%%%%%%%%%%%%%%%%%%%%%%%
%% 4. Bibliography 
%% Patches tabularx/xltabular, so it must come AFTER them
\usepackage{tabularx}
\usepackage{biblatex}

%%%%%%%%%%%%%%%%%%%%%%%%%%%%%%%%%%%%%%%%%%
%% 2. Tables 
%% MUST load before biblatex to avoid "Patching 'tabularx' failed" error
\usepackage{xltabular} 

%%%%%%%%%%%%%%%%%%%%%%%%%%%%%%%%%%%%%%%%%%
%% 3. This Package
\usepackage{aidisclose}

%%%%%%%%%%%%%%%%%%%%%%%%%%%%%%%%%%%%%%%%%%
%% 5. Hyperlinks
%% Almost always goes LAST
\usepackage[colorlinks,allcolors=blue!70!black]{hyperref}



%%%%%%%%%%%%%%%%%%%%%%%%%%%%%%%%%%%%%%%%%%
%% MAIN DOCUMENT
\begin{document}

\title{The \texttt{aidisclose} package}
\author{João M. Lourenço\\\url{https://github.com/joaomlourenco/aidisclose}}
\date{2025-12-26~(\gaiversion)}
\maketitle

\section{Introduction}
\label{sec:introduction}

\texttt{aidisclose} is a \LaTeX\ package providing a standardized, transparent mechanism for declaring the use of Generative Artificial Intelligence (GAI) tools in academic, technical, and professional documents.

The package is designed to support emerging ethical, institutional, and publisher requirements concerning AI-assisted content creation. It implements the \emph{GAIDeT (Generative AI Delegation Taxonomy)}, as proposed by \citeauthor{Suchikova:2025:GAIDeT}~\citeyear{Suchikova:2025:GAIDeT}~\cite{Suchikova:2025:GAIDeT}.

This package allows authors to:

\begin{itemize}
\item Select specific tasks delegated to AI from the predefined taxonomy (e.g., idea generation, data cleaning, text summarization).
\item List the specific GAI tools used (e.g., ChatGPT, Gemini, Claude).
\item Add optional explanatory comments (numbered or unnumbered).
\item Automatically generate a formatted "Disclosure of Delegation to Generative AI" section, including the necessary responsibility statements.
\item Optionally skip automatic section title generation for custom document structures.
\end{itemize}

\section{Usage}
\label{sec:usage}

To use the package, load it in your document preamble:

\begin{verbatim}
\usepackage{aidisclose}
\end{verbatim}

The declaration process consists of two steps: \textbf{Configuration} (defining what was done) and \textbf{Rendering} (printing the declaration).

\subsection{Configuration}

Configuration commands can be placed in the preamble or anywhere in the document body before the rendering command is called.

\subsubsection{Activating Taxonomy Items}

Use the \verb|\GAIactivate{}| command to check specific items in the taxonomy. A full list of available keys is provided in Section~\ref{sec:taxonomy-keys}.

\begin{verbatim}
% Example: Activating "Idea generation" and "Code optimization"
\GAIactivate{c:idea}
\GAIactivate{s:opt}
\end{verbatim}

\subsubsection{Specifying Tools}

Use the \verb|\GAItoolsUsed{}| command to list the AI tools employed. The package automatically formats the output sentence based on the number of tools provided (handling singular/plural forms and commas).

\begin{verbatim}
% Example 1: No tools used
\GAItoolsUsed{}

% Example 2: One tool
\GAItoolsUsed{ChatGPT-4}

% Example 3: Multiple tools
\GAItoolsUsed{ChatGPT-4, Gemini Advanced, Claude 3}
\end{verbatim}

\subsubsection{Adding Comments}

The package provides two comment environments: \texttt{GAIcomment} (numbered) and \texttt{GAIcomment*} (unnumbered).

\begin{verbatim}
% Numbered comment
\begin{GAIcomment}
The AI was used primarily for refining the code in Section 3,
but the algorithm logic was derived manually.
\end{GAIcomment}

% Unnumbered comment
\begin{GAIcomment*}
No GAI tools were used for data analysis.
\end{GAIcomment*}
\end{verbatim}

\subsubsection{Customizing the Title}

You can change the default title of the declaration section using \verb|\GAIdiscloseTitle|.

\begin{verbatim}
\GAIdiscloseTitle[Short Title]{Full Title for the Section}[section-level]
\end{verbatim}

The last (optional) argument \verb|section-level| defaults to \verb|\chapter| if \verb|\chapter| is defined (e.g., in book-like documents), otherwise defaults to \verb|\section|.

You can retrieve the stored titles using:

\begin{verbatim}
\GAIdiscloseTitleLong   % Full title
\GAIdiscloseTitleShort  % Short title
\end{verbatim}

\subsubsection{Customizing the Checkbox Symbol}

You can change the symbol used inside checkboxes with \verb|\GAIsetCheckmarkSymbol{}|. The package automatically recalculates box width to ensure proper alignment.

\begin{verbatim}
\GAIsetCheckmarkSymbol{\checkmark}  % Default
\GAIsetCheckmarkSymbol{\texttimes}  % Use × instead
\end{verbatim}

\subsubsection{Customizing the Font Size}

Change the font size used for the checklist:

\begin{verbatim}
\GAIsetChecklistFontSize{\small}
\end{verbatim}

The default is \verb|\smaller|, which means ``slightly smaller than current font size''.

\subsection{Rendering the Declaration}

To print the final declaration, use the \verb|\GAIrenderDeclaration| command at the desired location in your document. For papers/articles, you may place it before or after the references. For theses/dissertations, please follow your institution's guidelines.

\begin{verbatim}
\GAIrenderDeclaration[<columns>]{<authors>}
\GAIrenderDeclaration*[<columns>]{<authors>}
\end{verbatim}

\begin{itemize}
\item \textbf{Star variant} (\verb|\GAIrenderDeclaration*|): Renders the disclosure \textbf{without} the automatic section title. Use this when you want to manage the heading yourself.

\item \textbf{Optional Argument} (\verb|[<columns>]|): The number of columns used for the checklist. The default is \texttt{3}.

\item \textbf{Mandatory Argument} (\verb|{<authors>}|): A comma-separated list of author names submitting the declaration.
\end{itemize}

\bigskip
\textbf{Examples:}

\begin{verbatim}
% Standard form with automatic title
\GAIrenderDeclaration[3]{Jane Doe, John Smith}

% Starred form without title (custom section)
\section*{AI Disclosure Statement}
\label{sec:ai-disclosure}
\GAIrenderDeclaration*[2]{Jane Doe, John Smith}

% Single column layout
\GAIrenderDeclaration[1]{Mary Johnson}
\end{verbatim}

\section{Taxonomy Keys}
\label{sec:taxonomy-keys}

The following keys correspond to the specific tasks defined in the GAIDeT taxonomy~\cite{Suchikova:2025:GAIDeT}. Use these keys with the \verb|\GAIactivate{}| command.

\begin{center}
\small
\begin{xltabular}{\textwidth}{ll}
\toprule
\textbf{Key} & \textbf{Description} \\
\midrule
\multicolumn{2}{c}{\textbf{Conceptualization}} \\
\texttt{c:idea} & Idea generation \\
\texttt{c:objective} & Defining the research objective \\
\texttt{c:rq} & Formulating research questions and hypotheses \\
\texttt{c:feas} & Feasibility assessment and risk evaluation \\
\texttt{c:pretest} & Preliminary hypothesis testing \\
\midrule
\multicolumn{2}{c}{\textbf{Literature Review}} \\
\texttt{l:search} & Literature search and systematization \\
\texttt{l:write} & Writing the literature review \\
\texttt{l:patents} & Analysis of market trends/patent environment \\
\texttt{l:gaps} & Novelty evaluation and gap identification \\
\midrule
\multicolumn{2}{c}{\textbf{Methodology}} \\
\texttt{m:design} & Research design \\
\texttt{m:protocols} & Experimental or research protocols \\
\texttt{m:methods} & Selection of research methods \\
\midrule
\multicolumn{2}{c}{\textbf{Software Development}} \\
\texttt{s:codegen} & Code generation \\
\texttt{s:opt} & Code optimization \\
\texttt{s:auto} & Process automation \\
\texttt{s:algs} & Algorithms for data analysis \\
\midrule
\multicolumn{2}{c}{\textbf{Data Management}} \\
\texttt{d:collect} & Data collection \\
\texttt{d:validate} & Validation \\
\texttt{d:clean} & Data cleaning \\
\texttt{d:curate} & Data curation and organization \\
\texttt{d:analyze} & Data analysis \\
\texttt{d:viz} & Visualization \\
\texttt{d:repro} & Reproducibility testing \\
\midrule
\multicolumn{2}{c}{\textbf{Writing and Editing}} \\
\texttt{w:textgen} & Text generation \\
\texttt{w:proof} & Proofreading and editing \\
\texttt{w:summarize} & Summarizing text \\
\texttt{w:concl} & Formulation of conclusions \\
\texttt{w:tone} & Adapting and adjusting emotional tone \\
\texttt{w:translate} & Translation \\
\texttt{w:reformat} & Reformatting \\
\texttt{w:press} & Press releases and outreach materials \\
\midrule
\multicolumn{2}{c}{\textbf{Ethics \& Supervision}} \\
\texttt{e:bias} & Bias analysis and discrimination assessment \\
\texttt{e:risk} & Ethical risk analysis \\
\texttt{e:compliance} & Compliance monitoring \\
\texttt{e:conf} & Data confidentiality monitoring \\
\texttt{sup:qa} & Quality assessment \\
\texttt{sup:trends} & Trend identification \\
\texttt{sup:limits} & Identification of limitations \\
\texttt{sup:recs} & Recommendations \\
\texttt{sup:pub} & Publication support \\
\bottomrule
\end{xltabular}
\end{center}

%—————————————————————————————————————————————————————————————————
% AI Disclosure Customization
%—————————————————————————————————————————————————————————————————

% --- Conceptualization ---
\GAIactivate{c:idea}        % Idea generation
\GAIactivate{c:objective}   % Defining the research objective
\GAIactivate{c:rq}          % Formulating research questions and hypotheses
\GAIactivate{c:feas}        % Feasibility assessment and risk evaluation
% \GAIactivate{c:pretest}   % Preliminary hypothesis testing

% --- Literature Review ---
\GAIactivate{l:search}      % Literature search and systematization
\GAIactivate{l:write}       % Writing the literature review
% \GAIactivate{l:patents}   % Market trends / patent environment
% \GAIactivate{l:gaps}      % Novelty evaluation and gap identification

% --- Methodology ---
\GAIactivate{m:design}      % Research design
% \GAIactivate{m:protocols} % Experimental / research protocols
% \GAIactivate{m:methods}   % Selection of research methods

% --- Software Development and Automation ---
% \GAIactivate{s:codegen}   % Code generation
% \GAIactivate{s:opt}       % Code optimization
\GAIactivate{s:auto}        % Process automation
\GAIactivate{s:algs}        % Algorithms for data analysis

% --- Data Management ---
% \GAIactivate{d:collect}   % Data collection
\GAIactivate{d:validate}    % Validation
% \GAIactivate{d:clean}     % Data cleaning
\GAIactivate{d:curate}      % Data curation and organization
\GAIactivate{d:analyze}     % Data analysis
% \GAIactivate{d:viz}       % Visualization
% \GAIactivate{d:repro}     % Reproducibility testing

% --- Writing and Editing ---
\GAIactivate{w:textgen}     % Text generation
% \GAIactivate{w:proof}     % Proofreading and editing
\GAIactivate{w:summarize}   % Summarizing text
% \GAIactivate{w:concl}     % Formulation of conclusions
\GAIactivate{w:tone}        % Emotional tone adjustment
% \GAIactivate{w:translate} % Translation
% \GAIactivate{w:reformat}  % Reformatting
\GAIactivate{w:press}       % Press releases and outreach materials

% --- Ethics Review / Supervision ---
\GAIactivate{e:bias}        % Bias analysis / discrimination assessment
\GAIactivate{e:risk}        % Ethical risk analysis
\GAIactivate{e:compliance}  % Compliance monitoring
% \GAIactivate{e:conf}      % Data confidentiality monitoring
% \GAIactivate{sup:qa}      % Quality assessment
% \GAIactivate{sup:trends}  % Trend identification
\GAIactivate{sup:limits}    % Identification of limitations
% \GAIactivate{sup:recs}    % Recommendations
% \GAIactivate{sup:pub}     % Publication support

% --- GAI tools used ---
% \GAItoolsUsed{}
% \GAItoolsUsed{ChatGPT-5.2}
\GAItoolsUsed{ChatGPT-5.2, Gemini 3 Pro, Claude 4.5 Opus, Perplexity 12.12.25}

% --- Optional extra note/comment ---
\begin{GAIcomment}
The AI was used primarily for refining the code in Section 3, but the algorithm logic was derived manually.
\end{GAIcomment}

\begin{GAIcomment}
This is a second additional comment!
\end{GAIcomment}

\begin{GAIcomment*}
This is a third additional comment! It is starred and, thus, does not get a number. Also, this comment is longer and, hopefully, will span to the next line!
\end{GAIcomment*}

\begin{GAIcomment}
This is the third non-starred additional comment!
\end{GAIcomment}

% --- Optional customization of title ---
% \GAIdiscloseTitle[Disclosure of Generative AI]
%                  {Disclosure of Generative Artificial Intelligence}[subsection]

%—————————————————————————————————————————————————————————————————
% End-of AI Disclosure Customization
%—————————————————————————————————————————————————————————————————

% Syntax: \GAIrenderDeclaration[checklist_ncolumns]{Author1 Name, Author2 Name, …}

% Example: Custom checkmark symbol
% \GAIsetCheckmarkSymbol{%
%   \rule{0pt}{1.6ex}%
%   \hspace{0.4ex}%
%   \raisebox{0.1ex}{\large \texttimes}%
%   \hspace{0.4ex}%
% }

\GAIrenderDeclaration[3]{Jane Doe, John Doe, Mary Doe}
% \GAIrenderDeclaration[2]{Jane Doe}

\printbibliography

\end{document}
