%% aidisclose-doc.tex --- Utilities for Generative AI disclosure checklist and statements
%%
%% Copyright (C) 2025-26 by João M. Lourenço <joao.lourenco@fct.unl.pt>
%%
%% This file may be distributed and/or modified under the conditions of
%% the LaTeX Project Public License, either version 1.3c of this license
%% or (at your option) any later version.
%%
%% The latest version of this license is in:
%%    http://www.latex-project.org/lppl.txt
%% and version 1.3c or later is part of all distributions of LaTeX
%% version 2006/05/20 or later.
%%
\documentclass[11pt]{article}

% \usepackage[portuguese]{babel} % Example: Uncomment to test language support

%%%%%%%%%%%%%%%%%%%%%%%%%%%%%%%%%%%%%%%%%%
%% 1. Basic Utilities & Formatting
\usepackage[T1]{fontenc}
\usepackage{xcolor}
\usepackage{url}
\usepackage{booktabs}
\usepackage{enumitem}
\usepackage{multicol}
\usepackage{minted}

%%%%%%%%%%%%%%%%%%%%%%%%%%%%%%%%%%%%%%%%%%
%% 2. Tables 
%% MUST load before biblatex to avoid "Patching 'tabularx' failed" error
\usepackage{tabularx}
\usepackage{xltabular} 

%%%%%%%%%%%%%%%%%%%%%%%%%%%%%%%%%%%%%%%%%%
%% 3. Bibliography 
\usepackage{biblatex}

%%%%%%%%%%%%%%%%%%%%%%%%%%%%%%%%%%%%%%%%%%
%% 4. This Package
%% Options: [autobib=true|false] (default: true)
\usepackage[autobib=true]{aidisclose}

%%%%%%%%%%%%%%%%%%%%%%%%%%%%%%%%%%%%%%%%%%
%% 5. Hyperlinks
\usepackage[colorlinks,allcolors=blue!70!black]{hyperref}

%%%%%%%%%%%%%%%%%%%%%%%%%%%%%%%%%%%%%%%%%%
%% 6. Customization for LaTeX pretty print
\definecolor{latexbg}{HTML}{F5F5EE}
\newminted{latex}{
  bgcolor=latexbg,
  fontsize=\small,
  breaklines,
  tabsize=2,
  autogobble,
  % frame=single,
  % framesep=3mm,
  % rulecolor=black!20,
}
\newmintinline{latex}{}
\newenvironment{ltx}
  {\VerbatimEnvironment
   \begin{minted}[
    bgcolor=latexbg,
    fontsize=\small,
    breaklines,
    tabsize=2,
    autogobble,
    % frame=single,
    % framesep=3mm,
    % rulecolor=black!20,
   ]{latex}}
  {\end{minted}}


\begin{document}

\title{The \texttt{aidisclose} package}
\author{João M. Lourenço\\\url{https://github.com/joaomlourenco/aidisclose}}
\date{2025-12-26~(\gaiversion)}
\maketitle

\begin{abstract}
    The \texttt{aidisclose} package provides a standardized, transparent mechanism for declaring the use of Generative Artificial Intelligence (GAI) tools in academic, technical, and professional documents. It implements the \emph{GAIDeT (Generative AI Delegation Taxonomy)}~\cite{Suchikova:2025:GAIDeT} and automates the creation of disclosure statements and checklists.
\end{abstract}


\tableofcontents

\section{Introduction}
\label{sec:introduction}

\texttt{aidisclose} is designed to support emerging ethical, institutional, and publisher requirements concerning AI-assisted content creation. It allows authors to:

\begin{itemize}
    \item Select specific tasks delegated to Generative ~AI (GAI) from the GAIDeT taxonomy~\cite{Suchikova:2025:GAIDeT} (e.g., idea generation, data cleaning, text summarization).
    \item List the specific GAI tools used (e.g., ChatGPT, Gemini, Claude).
    \item Add optional explanatory comments (numbered or unnumbered).
    \item Automatically generate a formatted “\emph{Disclosure of Delegation to Generative AI}” section/chapter.
    \item Automatically handle citations for the taxonomy and the package itself.
\end{itemize}

\section{Package Loading and Options}
\label{sec:loading}

Load the package in your document preamble:

\begin{ltx}
\usepackage[<options>]{aidisclose}
\end{ltx}

\subsection{Options}
The package currently supports the following key-value option:

\begin{description}
    \item[\texttt{autobib=true|false}] (Default: \texttt{true})\\
    When enabled, the package automatically:
    \begin{enumerate}
        \item Writes a \texttt{aidisclose.bib} file containing the references for GAIDeT~\cite{Suchikova:2025:GAIDeT} and this package~\cite{Lourenco:2025:aidisclose}.
        \item Loads this bibliography resource (compatible with \texttt{biblatex} and standard \texttt{BibTeX}).
    \end{enumerate}
    Set this to \texttt{false} if you wish to manage these citations manually in your own \texttt{.bib} file.
\end{description}

\section{Internationalization}
\label{sec:i18n}

The package automatically detects the document language (via \texttt{babel} or \texttt{polyglossia}) and loads the corresponding translation file (\texttt{.ldf}).

\paragraph{Currently (v\gaiversion) Supported Languages:}~

\begin{itemize}[
    itemsep = 1\itemsep,
    parsep = 1\parsep,
    before = \raggedcolumns\begin{multicols}{3}\raggedright,
    after = \end{multicols}
  ]
    \item English (\texttt{en}) \\— Default
    \item Portuguese (\texttt{pt})
    \item Spanish (\texttt{es})
    \item French (\texttt{fr})
    \item German (\texttt{de})
    \item Italian (\texttt{it})
    \item Dutch (\texttt{nl})
    \item Danish (\texttt{dk})
    \item Greek (\texttt{gr})
    \item Czech (\texttt{cz})
    \item Polish (\texttt{pl})
    \item Slovak (\texttt{sk})
    \item Ukrainian (\texttt{uk})
    \item Catalan (\texttt{cat})
\end{itemize}

If the detected language is not supported, the package falls back to English.

\section{Usage}
\label{sec:usage}

The declaration process consists of two steps: \textbf{Configuration} (defining what was done) and \textbf{Rendering} (printing the declaration).

\subsection{Configuration}

Configuration commands can be placed in the preamble or in the document body before the rendering command is called.

\subsubsection{Activating Taxonomy Items}
Use \latexinline|\GAIactivate{}| to check specific items in the taxonomy. See Section~\ref{sec:taxonomy-keys} for all available keys.

\begin{ltx}
% Example: Activating "Idea generation" and "Code optimization"
\GAIactivate{c:idea}
\GAIactivate{s:opt}
\end{ltx}

\subsubsection{Specifying Tools}
Use \latexinline|\GAItoolsUsed{}| to list the AI tools employed. The package handles formatting (singular/plural) automatically.

\begin{ltx}
% Example 1: No tools used
\GAItoolsUsed{}

% Example 2: Multiple tools
\GAItoolsUsed{ChatGPT-4, Gemini Advanced, Claude 3}
\end{ltx}

\subsubsection{Adding Comments}
Use the \texttt{GAIcomment} (numbered) and \texttt{GAIcomment*} (unnumbered) environments for details.
Comments may contain multiple paragraphs.

\begin{ltx}
\begin{GAIcomment}
The AI was used primarily for refining the code in Section 3.
\end{GAIcomment}

\begin{GAIcomment*}
No GAI tools were used for data analysis.
\end{GAIcomment*}
\end{ltx}

\subsubsection{Customizing the Title}
Change the default section title and hierarchy level using \latexinline|\GAIdiscloseTitle|.

\begin{ltx}
\GAIdiscloseTitle[Short Title]{Full Title}[section-level]
\end{ltx}

\begin{itemize}
    \item \textbf{section-level}: Defaults to \latexinline|\chapter| if defined, otherwise \latexinline|\section|.
\end{itemize}

\subsubsection{Visual Customization}

\begin{itemize}
    \item \textbf{Checkmark Symbol:} \latexinline|\GAIsetCheckmarkSymbol{\texttimes}| (default is \latexinline|\checkmark|).
    \item \textbf{Font Size:} \latexinline|\GAIsetChecklistFontSize{\small}| (default is \latexinline|\smaller|, meaning: \emph{slightly smaller than the current font size}).
\end{itemize}

\subsection{Rendering the Declaration}

Place the \latexinline|\GAIrenderDeclaration| command where you want the disclosure to appear (e.g., after the Conclusion or before References).

\begin{ltx}
\GAIrenderDeclaration[<columns>]{<authors>}
\GAIrenderDeclaration*[<columns>]{<authors>}
\end{ltx}

\begin{itemize}
    \item \textbf{Star variant (*):} Renders the checklist \emph{without} the section heading.
    \item \textbf{<columns>:} Number of columns for the checklist (default: 3).
    \item \textbf{<authors>:} Comma-separated list of (document) authors declaring the use of AI.
\end{itemize}

\section{Taxonomy Keys}
\label{sec:taxonomy-keys}

Use these keys with \latexinline|\GAIactivate{}|. Derived from the GAIDeT taxonomy~\cite{Suchikova:2025:GAIDeT}.

\begin{center}
\small
\begin{xltabular}{\textwidth}{ll}
\toprule
\textbf{Key} & \textbf{Description} \\
\midrule
\multicolumn{2}{c}{\textbf{Conceptualization}} \\
\texttt{c:idea} & Idea generation \\
\texttt{c:objective} & Defining the research objective \\
\texttt{c:rq} & Formulating research questions and hypotheses \\
\texttt{c:feas} & Feasibility assessment and risk evaluation \\
\texttt{c:pretest} & Preliminary hypothesis testing \\
\midrule
\multicolumn{2}{c}{\textbf{Literature Review}} \\
\texttt{l:search} & Literature search and systematization \\
\texttt{l:write} & Writing the literature review \\
\texttt{l:patents} & Analysis of market trends/patent environment \\
\texttt{l:gaps} & Novelty evaluation and gap identification \\
\midrule
\multicolumn{2}{c}{\textbf{Methodology}} \\
\texttt{m:design} & Research design \\
\texttt{m:protocols} & Experimental or research protocols \\
\texttt{m:methods} & Selection of research methods \\
\midrule
\multicolumn{2}{c}{\textbf{Software Development}} \\
\texttt{s:codegen} & Code generation \\
\texttt{s:opt} & Code optimization \\
\texttt{s:auto} & Process automation \\
\texttt{s:algs} & Algorithms for data analysis \\
\midrule
\multicolumn{2}{c}{\textbf{Data Management}} \\
\texttt{d:collect} & Data collection \\
\texttt{d:validate} & Validation \\
\texttt{d:clean} & Data cleaning \\
\texttt{d:curate} & Data curation and organization \\
\texttt{d:analyze} & Data analysis \\
\texttt{d:viz} & Visualization \\
\texttt{d:repro} & Reproducibility testing \\
\midrule
\multicolumn{2}{c}{\textbf{Writing and Editing}} \\
\texttt{w:textgen} & Text generation \\
\texttt{w:proof} & Proofreading and editing \\
\texttt{w:summarize} & Summarizing text \\
\texttt{w:concl} & Formulation of conclusions \\
\texttt{w:tone} & Adapting and adjusting emotional tone \\
\texttt{w:translate} & Translation \\
\texttt{w:reformat} & Reformatting \\
\texttt{w:press} & Press releases and outreach materials \\
\midrule
\multicolumn{2}{c}{\textbf{Ethics \& Supervision}} \\
\texttt{e:bias} & Bias analysis and discrimination assessment \\
\texttt{e:risk} & Ethical risk analysis \\
\texttt{e:compliance} & Compliance monitoring \\
\texttt{e:conf} & Data confidentiality monitoring \\
\texttt{sup:qa} & Quality assessment \\
\texttt{sup:trends} & Trend identification \\
\texttt{sup:limits} & Identification of limitations \\
\texttt{sup:recs} & Recommendations \\
\texttt{sup:pub} & Publication support \\
\bottomrule
\end{xltabular}
\end{center}

%—————————————————————————————————————————————————————————————————
% Example Usage
%—————————————————————————————————————————————————————————————————

\section{Example Output}

Below is an example of a rendered declaration. The configuration used is:

\begin{ltx}
% Configuration
\GAIactivate{c:idea}
\GAIactivate{c:rq}
\GAIactivate{l:search}
\GAIactivate{l:write}
\GAIactivate{s:auto}
\GAIactivate{d:analyze}
\GAIactivate{w:summarize}
\GAIactivate{w:reformat}

\GAItoolsUsed{ChatGPT-4o, Gemini 1.5 Pro, GitHub Copilot}

\begin{GAIcomment}
AI was used for refining code structure in Section~4.
\end{GAIcomment}
\end{ltx}

% --- Conceptualization ---
\GAIactivate{c:idea}        % Idea generation
% \GAIactivate{c:objective}   % Defining the research objective
\GAIactivate{c:rq}          % Formulating research questions and hypotheses
% \GAIactivate{c:feas}        % Feasibility assessment and risk evaluation
% \GAIactivate{c:pretest}     % Preliminary hypothesis testing

% --- Literature Review ---
\GAIactivate{l:search}      % Literature search and systematization
\GAIactivate{l:write}       % Writing the literature review
% \GAIactivate{l:patents}     % Market trends / patent environment
% \GAIactivate{l:gaps}        % Novelty evaluation and gap identification

% --- Methodology ---
% \GAIactivate{m:design}      % Research design
% \GAIactivate{m:protocols}   % Experimental / research protocols
% \GAIactivate{m:methods}     % Selection of research methods

% --- Software Development and Automation ---
% \GAIactivate{s:codegen}     % Code generation
% \GAIactivate{s:opt}         % Code optimization
\GAIactivate{s:auto}        % Process automation
% \GAIactivate{s:algs}        % Algorithms for data analysis

% --- Data Management ---
% \GAIactivate{d:collect}     % Data collection
% \GAIactivate{d:validate}    % Validation
% \GAIactivate{d:clean}       % Data cleaning
% \GAIactivate{d:curate}      % Data curation and organization
\GAIactivate{d:analyze}     % Data analysis
% \GAIactivate{d:viz}         % Visualization
% \GAIactivate{d:repro}       % Reproducibility testing

% --- Writing and Editing ---
% \GAIactivate{w:textgen}     % Text generation
% \GAIactivate{w:proof}       % Proofreading and editing
\GAIactivate{w:summarize}   % Summarizing text
% \GAIactivate{w:concl}       % Formulation of conclusions
% \GAIactivate{w:tone}        % Emotional tone adjustment
% \GAIactivate{w:translate}   % Translation
\GAIactivate{w:reformat}    % Reformatting
% \GAIactivate{w:press}       % Press releases and outreach materials

% --- Ethics Review / Supervision ---
% \GAIactivate{e:bias}        % Bias analysis / discrimination assessment
% \GAIactivate{e:risk}        % Ethical risk analysis
% \GAIactivate{e:compliance}  % Compliance monitoring
% \GAIactivate{e:conf}        % Data confidentiality monitoring
% \GAIactivate{sup:qa}        % Quality assessment
% \GAIactivate{sup:trends}    % Trend identification
% \GAIactivate{sup:limits}    % Identification of limitations
% \GAIactivate{sup:recs}      % Recommendations
% \GAIactivate{sup:pub}       % Publication support


\begin{GAIcomment}
AI was used for refining code structure in Section~4.
\end{GAIcomment}

% Rendering
\clearpage
\GAIrenderDeclaration[3]{Jane Doe, John Smith}

\printbibliography

\end{document}